% --------------------------------------------------------------
% This is all preamble stuff that you don't have to worry about.
% Head down to where it says "Start here"
% --------------------------------------------------------------
 
\documentclass[12pt]{article}
 
\usepackage[margin=1in]{geometry} 
\usepackage{amsmath,amsthm,amssymb,scrextend}
\usepackage{fancyhdr}
\pagestyle{fancy}

 
\newcommand{\N}{\mathbb{N}}
\newcommand{\Z}{\mathbb{Z}}
\newcommand{\I}{\mathbb{I}}
\newcommand{\R}{\mathbb{R}}
\newcommand{\Q}{\mathbb{Q}}
\renewcommand{\qed}{\hfill$\blacksquare$}
\let\newproof\proof
\renewenvironment{proof}{\begin{addmargin}[1em]{0em}\begin{newproof}}{\end{newproof}\end{addmargin}\qed}
% \newcommand{\expl}[1]{\text{\hfill[#1]}$}
 
\newenvironment{theorem}[2][Theorem]{\begin{trivlist}
\item[\hskip \labelsep {\bfseries #1}\hskip \labelsep {\bfseries #2.}]}{\end{trivlist}}
\newenvironment{lemma}[2][Lemma]{\begin{trivlist}
\item[\hskip \labelsep {\bfseries #1}\hskip \labelsep {\bfseries #2.}]}{\end{trivlist}}
\newenvironment{problem}[2][Problem]{\begin{trivlist}
\item[\hskip \labelsep {\bfseries #1}\hskip \labelsep {\bfseries #2.}]}{\end{trivlist}}
\newenvironment{exercise}[2][Exercise]{\begin{trivlist}
\item[\hskip \labelsep {\bfseries #1}\hskip \labelsep {\bfseries #2.}]}{\end{trivlist}}
\newenvironment{reflection}[2][Reflection]{\begin{trivlist}
\item[\hskip \labelsep {\bfseries #1}\hskip \labelsep {\bfseries #2.}]}{\end{trivlist}}
\newenvironment{proposition}[2][Proposition]{\begin{trivlist}
\item[\hskip \labelsep {\bfseries #1}\hskip \labelsep {\bfseries #2.}]}{\end{trivlist}}
\newenvironment{corollary}[2][Corollary]{\begin{trivlist}
\item[\hskip \labelsep {\bfseries #1}\hskip \labelsep {\bfseries #2.}]}{\end{trivlist}}
 
\begin{document}
 
% --------------------------------------------------------------
%                         Start here
% --------------------------------------------------------------

\lhead{Math 300}
\chead{Chesyti Brown}
\rhead{\today}
 
% \maketitle
 
\begin{theorem}{2.3} %You can use theorem, proposition, exercise, o reflection here.  Modify x.yz to be whatever number you are proving
DeMorgan's Laws for sets. Let A and B be sets. Then we have

1. $\overline{A \cup B}$ = $\overline{A} \cap \overline{B}$

2. $\overline{A \cap B}$ = $\overline{A} \cup \overline{B}$

\end{theorem} 

\begin{proof}


	To prove that $\overline{A \cup B}$ = $\overline{A} \cap \overline{B}$ , we start by showing that each set is a subset of the other.The definition of a subset states that A is a subset of B if every element $a \in A$ is also an element of B. Since A and B are sets, if $A \subset B$ and $B \subset A$, then A = B. \\
	
    Suppose $x \in \overline{A \cup B}$ , which means $x \notin A \cup B$. Then $x \notin A$ and $x \notin B$. Hence, $x \in \overline{A}$ and $x \in \overline{B}$. This means $x \in \overline{A \cap B}$. Thus, $\overline{A \cup B} \subset \overline{A} \cap \overline{B}$. Now suppose, $x \in \overline{A} \cap \overline{B}$. Then $x \in \overline{A}$ and $x \in \overline{B}$. Hence $x \notin A$ and $x \notin B$, which means that $x \notin A \cup B$. Therefore, $x \in \overline{A \cup B}$. Thus proving that $\overline{A \cup B}$ = $\overline{A} \cap \overline{B}$.\\

	To prove that $\overline{A \cap B}$ = $\overline{A} \cup \overline{B}$ , we start by showing that each set is a subset of the other. Suppose $x \in \overline{A \cap B}$ , which means $x \notin A \cap B$. Then $x \notin A$ and $x \notin B$. Hence, $x \in \overline{A}$ and $x \in \overline{B}$. This means $x \in \overline{A \cup B}$. Thus, $\overline{A \cap B} \subset \overline{A} \cup \overline{B}$. Now suppose, $x \in \overline{A} \cup \overline{B}$. Then $x \in \overline{A}$ and $x \in \overline{B}$. Hence $x \notin A$ and $x \notin B$, which means that $x \notin A \cap B$. Therefore, $x \in \overline{A \cap B}$. Thus proving that $\overline{A \cap B}$ = $\overline{A} \cup \overline{B}$.
%Note 1: The * tells LaTeX not to number the lines.  If you remove the *, be sure to remove it below, too.
%Note 2: Inside the align environment, you do not want to use $-signs.  The reason for this is that this is already a math environment. This is why we have to include \text{} around any text inside the align environment.

\end{proof}
 
\begin{theorem}{4.4}
Let a,b and c $\in \mathbb{Z}$. If a divides b and b divides c then a divides c.


\end{theorem}
 
\begin{proof}[Proof]%Whatever you put in the square brackets will be the label for the block of text to follow in the proof environment.
Assume a divides b and b divides c. Since a divides b, there exists $n_1$ $\in \mathbb{Z}$ such that a$n_1$=b. Since b divides c, there exists $n_2$ such that b$n_2$=c.Since we know the existential statement is true in the universe you can use it to create an instance of an object with the property it describes. So, we let m=$n_1n_2$.Then \\

					am= {$an_1n_2$} = {$bn_2$} = c \\

Since am=c, we have shown that a divides c.

\end{proof}

\begin{theorem}{7.11}
Suppose that R is a relation on A. Then R is both symmetric and anti-symmetric, if and only if $R \subset $ $Id_A$.

\end{theorem}


\begin{proof}[Proof]
	
    
    Assume R is symmetric and anti-symmetric. This means that all ordered pairs (a,b) $\in$ R, there must be a pair (b,a) $\in$ R, and this can only be true when a=b. This means every ordered pair in R is a value a$\in$ A relates to itself. Since the $Id_A$ is a relation that includes every value in A related to itself, R must be a subset of $Id_A$. 
	
    
    Assume R $\subset$ $Id_A$. This means that R can only have elements that are also in $Id_A$. Therefore every element of R is an ordered pair (a,b) $\in$ A where a=b. Since a=b in every element of R, it satisfied the conditions for anti-symmetry. Also, Since (a,b) = (b,a) , R also satisfies the conditions for symmetry. 
  

\end{proof}


\begin{theorem}{10.9}

Let x $\neq$ 1 be any real number. For all natural numbers n we have 
	
$\frac{x^{n}-1}{x-1}$ = ${x^{n-1}}$+${x^{n-2}}$+...+$x^{2}$+$x$+1

\end{theorem}

\begin{proof}[Proof]
First, we define set S as the set {n$\in$ $\mathbb{N}$} such that $\sum_{k=1}^{n}x^{k-1}= \frac{x^{n-1}}{x-1}$ 

I will induct on $n$ \\
\textbf{Base Case (n=1): } $\sum_{k=1}^{1}x^{k-1} = \frac{x^{1}-1}{x-1}=1$\\

\textbf{Inductive Hypothesis: } Assume $\sum_{k=1}^{n}x^{k-1}= \frac{x^{n-1}}{x-1}$ holds for n. \\

\textbf{Inductive Step: } We want to show $\sum_{k=1}^{n+1}x^{k-1}= \frac{x^{n+1}}{x-1}$ \\
$=\frac{x^{n+1}-x^{n}+x^{n-1}}{x-1}$\\
$=\frac{x(x^{n})-x^{n}+x^{n}-1}{x-1}$\\
$=\frac{x(x^{n})}{x-1}$-$\frac{x^{n}}{x-1}$+$\frac{x^{n-1}}{x-1}$\\
=$\sum_{k=1}^{n}x^{k-1}$+$\frac{x(x^{n})}{x-1}$-$\frac{x^{n}}{x-1}$\\
=$\sum_{k=1}^{n}x^{k-1}$+$x^{n}$\\
=$1$ + $x^{1}$ + $x^{2}$ + ... +$x^{n-2}$+$x^{n-1}$+$x^{n}$\\
=$\sum_{k=1}^{n+1}x^{k-1}$\\
Therefore, $S$=$\mathbb{N}$.

\end{proof}
 
\begin{theorem}{3.2}
If $\mathcal{F}$ is a family of sets and A $\in$ $\mathcal{F}$ then A $\subset$ $\cup$$\mathcal{F}$.

\end{theorem}

\begin{proof}[Proof]
Let x$\in$ A be arbitrary.We know A$\in$$\mathcal{F}$.Therefore, we know that there exists A $\in$$\mathcal{F}$ such that x$\in$ A. Therefore by definition of $\cup$$\mathcal{F}$, which states that the union of $\mathcal{F}$ is the collections of all sets that are elements of $\mathcal{F}$ where there exists x$\in$A for some A$\in$$\mathcal{F}$, x $\in$$\cup$$\mathcal{F}$. Since x is arbitrary, we have shown that A $\subset$ $\cup$$\mathcal{F}$.

\end{proof}

\begin{theorem}{6.1}
If $0< a < b$ where a and b are real numbers, then $a^{2} < b^{2}$.


\end{theorem}

\begin{proof}[Proof]
We assume that $b > a > 0$. Since $a > 0$, if we multiply both sides of $b > a$ by a, we get $ab > a^{2}$. Similarly, since $b > 0$, if we multiply both sides of $b > a$ by b, we get $b^{2} > ab$. We can combine $ab > a^{2}$ and $b^{2} > ab$ to get $b^{2} > ab > a^{2}$. Therefore by the transitive property, $b^{2} > a^{2}$. Thus we have shown that if $b > a > 0$, then $b^{2} > a^{2}$.

\end{proof}


% --------------------------------------------------------------
%     You don't have to mess with anything below this line.
% --------------------------------------------------------------
 


\end{document}