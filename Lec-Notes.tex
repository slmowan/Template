\documentclass{article}

\usepackage[latin1]{inputenc}
\usepackage{color}
\usepackage{listings}
\usepackage[english]{babel}
\usepackage{graphicx}
\usepackage{indentfirst}

%\usepackage{fontspec}
%\usepackage[utf8]{inputenc}
%\setmainfont{Futura}[ItalicFont={Futura Italic}]

\definecolor{colKeys}{rgb}{0,0,1}
\definecolor{colIdentifier}{rgb}{0,0,0}
\definecolor{colComments}{rgb}{0,0.5,1}
\definecolor{colString}{rgb}{0.6,0.1,0.1}

\lstset{%configuration de listings
float=hbp,
basicstyle=\ttfamily\small, 
identifierstyle=\color{colIdentifier}, 
keywordstyle=\color{colKeys}, 
stringstyle=\color{colString}, 
commentstyle=\color{colComments}, 
columns=flexible, 
tabsize=2, 
%frame=trBL, 
frameround=tttt, 
extendedchars=true, 
showspaces=false, 
showstringspaces=false, 
numbers=left, 
numberstyle=\tiny, 
breaklines=true, 
breakautoindent=true, 
captionpos=b,
xrightmargin= 1cm, 
xleftmargin= 1cm
} 
\lstset{language=c++}
\lstset{commentstyle=\textit}


\parskip 5pt plus 2pt minus 2pt
\textwidth=14cm
\oddsidemargin=1cm
\evensidemargin=1cm
\topmargin=-0.5cm
\headheight=0cm
\headsep=1cm
\textheight=23cm
\parindent=1.5cm

\begin{document}
\thispagestyle{empty}
\begin{center}
\fbox{\large\bf Brief description of Gaussian Integral and its characteristics that make it unique }
\end{center}
\bigskip


\section*{1. INTRODUCTION}

\noindent 1.1) What is a Guassian Integral?
The Gaussian integral, which is also known as the Euler Poisson integral, is the integral of the Gaussian function $e^{-x2}$ over the entire real line . It is named after the German mathematician Carl Friedrich Gauss. The integral is:
\begin{equation}
 \displaystyle{ \int_{ -\infty}^{\infty} e^{-x^2} dx = \sqrt{\pi}} 
\end{equation}
The Gaussian integral, is  probability integral and closely related to the erf function, it is the integral of the one-dimensional Gaussian function over $(-\infty,\infty)$. It can be computed using the trick of combining two one-dimensional Gaussians

\noindent 1.2) How it can be proved ?
The Gaussain integral can be proved in many ways (Polar coordinates,Another differentiation under integral sign,A volume integral,Asymptotic estimates,Stirling$^{'}$s formula etc..).
\section*{2. Some Characteristics of Gaussian Integral}

\noindent 2.1.0) The Guassian integral appears in many situation in mathematics and statistics and has a wide range of application . Gaussian integral is exactly the the reason why the factor ${\frac{1}{2\pi}}$ is require in the normalization (Gaussian Distribution) of all the probabilities . 

\noindent 2.1.1) With a slight change of variables it is used to compute the normalizing constant of the normal distribution 



\noindent 2.1.2) The same integral with finite limits is closely related to both the error function and the cumulative distribution function of the normal distribution


\noindent 2.1.3) Used for choosing an appropriate coordinate system to describe the given volume V or surface S

\noindent 2.1.4) Finding the limits on the integrals in this system in terms of the variables of integration chosen

\noindent 2.1.5) Expressing the integrand and volume element dV or surface element dS in terms of the differentials of the chosen variable

\section*{3. Why Gaussian distribution is important}
\noindent The Gaussian distribution or normal distribution sometimes informallycalled as $"$Bell curve$"$ is the most important countinuos distribution in probability and it has a wide range of applications.The normal distribution is important because of its $"$ Central Limit Theorem$"$.The four main characteristics of a normal distribution are symmetric,unimodal,and asymptotic,and the mean,median,and mode are all equal.
 
\end{document}